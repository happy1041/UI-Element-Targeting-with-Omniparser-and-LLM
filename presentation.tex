\documentclass[12pt, aspectratio=169, xetex]{beamer}

% --- 基础设置 ---
\usepackage{xeCJK} % 支持中文(XeLaTeX 环境下无需 [UTF8] 选项)
\usetheme{Madrid}         % 使用自带的 Madrid 主题,兼容性最高
\usecolortheme{whale}     % 使用深蓝色调(稍后手动调红)

% 自定义北大红颜色
\definecolor{PKURed}{RGB}{139,0,0} 
\setbeamercolor{structure}{fg=PKURed}
\setbeamercolor{titlelike}{parent=structure,bg=white}
\setbeamercolor{footline}{bg=PKURed,fg=white}
\setbeamercolor{author in head/foot}{bg=PKURed!80!black,fg=white}
\setbeamercolor{title in head/foot}{bg=PKURed!60!black,fg=white}
\setbeamercolor{date in head/foot}{bg=PKURed!40!black,fg=white}

% --- 标题信息 ---
\title{OmniParser + LLM}
\subtitle{UI 智能体技术实现与视觉定位优化汇报}
\author{赵亦阳,弥梓睿,钟骏宇}
\date{\today}
\institute{PKU AI4S Winter Camp}

\begin{document}

% 标题页
\maketitle

% 目录页
\begin{frame}{目录}
    \tableofcontents
\end{frame}

% --- 第1部分:项目概述 ---
\section{项目概述}
\begin{frame}{项目概述}
    \begin{itemize}
        \item \textbf{核心目标}: 构建能够理解自然语言指令并精准定位 UI 元素的智能体。
        \item \textbf{核心痛点}: 
        \begin{itemize}
            \item 传统 UI 自动化定位不稳定。
            \item 纯文本 OCR 缺失视觉语义。
            \item 绝对坐标生成容易偏移。
        \end{itemize}
        \item \textbf{解决方案}: 视觉解析解析引擎 (OmniParser) + 多模态大脑 (Gemini-3.0-flash)。
    \end{itemize}
\end{frame}

% --- 第2部分:技术架构 ---
\section{技术栈核心}
\begin{frame}{双引擎推理架构}
    \begin{columns}
        \begin{column}{0.5\textwidth}
            \textbf{视觉解析 (OmniParser)}
            \begin{itemize}
                \item \textbf{YOLOv8}: 检测图标 (Icon) 与 UI 元素。
                \item \textbf{Florence-2}: 为检测元素生成语义描述 (Captioning)。
            \end{itemize}
        \end{column}
        \begin{column}{0.5\textwidth}
            \textbf{逻辑推理 (LLM)}
            \begin{itemize}
                \item \textbf{Gemini-3.0-flash}: 跨模态理解。
                \item \textbf{视觉推断}: 结合原始图像与元素元数据列表进行交互决策。
            \end{itemize}
        \end{column}
    \end{columns}
\end{frame}

% --- 第3部分:核心创新点 ---
\section{核心创新与优化}

\begin{frame}{核心创新 1: 视觉优先的 ID 决策机制}
    \begin{itemize}
        \item \textbf{实现逻辑}: 弃用 LLM 直接预测坐标,改为“预标记 + ID 选定”。
        \item \textbf{优势}: 
        \begin{itemize}
            \item 强制 LLM 在已有的检测库中进行比对。
            \item 实现了 BBOX 的准确召回。
            \item 显著降低了幻觉 (Hallucination) 导致的点击偏移。
        \end{itemize}
    \end{itemize}
\end{frame}

\begin{frame}{核心创新 2: OCR 视觉偏见纠正方案}
    \begin{itemize}
        \item \textbf{背景}: OCR 常将图标误识别为字符(如“三”代表菜单、“X”代表关闭)。
        \item \textbf{优化手段}:
        \begin{itemize}
            \item 在 Prompt 中加入 \textbf{OCR Correction Logic}。
            \item 强制模型进行 [视觉验证] vs [OCR 结果] 的冲突检测。
            \item 提升了在 Office、Adobe 等生产力工具中的指令召回率。
        \end{itemize}
    \end{itemize}
\end{frame}

\begin{frame}{核心创新 3: 稠密 UI 场景的动态参数调节}
    针对图标极其密集的专业软件(如 Word),网页端提供了参数修正功能:
    \begin{itemize}
        \item \textbf{IOU Threshold}: 调高——防止相邻小图标被归并为一个大框。
        \item \textbf{Detect Image Size}: 增大——提升对极小按钮的识别精度。
        \item \textbf{Box Threshold}: 下调——有助于找回被漏检的功能组件。
    \end{itemize}
\end{frame}

\begin{frame}{核心创新 4: 规范化中间过程(JSON 解析系统)}
    \begin{itemize}
        \item \textbf{鲁棒性}: 应对 LLM 返回格式多变(\textit{target\_id, matched\_id} 等)的问题。
        \item \textbf{机制}:
        \begin{itemize}
            \item 自动纠错字段名。
            \item 递归/正则提取嵌套 ID。
            \item 坏结果自修复,确保 Gradio 界面零延迟零崩溃。
        \end{itemize}
    \end{itemize}
\end{frame}

% --- 第4部分:UI 功能展示 ---
\section{UI 系统与交互设计}
\begin{frame}{开发者导向的交互界面}
    \begin{itemize}
        \item \textbf{Omni-View}: 提供全量 UI element 源数据列表,透明化黑盒过程。
        \item \textbf{状态追踪}: 完整显示推理过程 (Reasoning) 。
    \end{itemize}
\end{frame}

% --- 第5部分:总结与规划 ---
\section{总结与展望}
\begin{frame}{总结与未来展望}
    \begin{columns}
        \begin{column}{0.5\textwidth}
            \textbf{主要成果}
            \begin{itemize}
                \item 成功处理日常应用中的 UI 和自然语言任务。
                \item 实现高容错性的图标交互。
            \end{itemize}
        \end{column}
        \begin{column}{0.5\textwidth}
            \textbf{未来规划}
            \begin{itemize}
                \item 多答案同时输出与验证机制。
                \item 支持 Step-by-Step 长链任务。
                \item 支持动画与 UI element 状态识别。
            \end{itemize}
        \end{column}
    \end{columns}
\end{frame}

\begin{frame}
    \vspace{2em}
    \begin{center}
        {\Huge \textcolor{PKURed}{谢谢观看!}}
        \vspace{1.5em}
        
        {\large Question \& Answer}
    \end{center}
\end{frame}

\end{document}
